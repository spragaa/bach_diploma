%!TEX root = ../thesis.tex
ШІ --- штучний інтелект

NIST - Національний інститут стандартів і технологій США (англ. U.S. National Institute of Standards and Technology)

RSA --- асиметрична криптосистема запропонована Рівестом, Шаміром та Адлеманом (англ. Rivest-Shamir-Adelman) у 1977 р.

AES --- покращений стандарт шифрування (англ. Advanced Encryption Standard) опублікований NIST  у 2001 р.

ГПВЧ --- генератор псевдовипадкових чисел

PCFG --- імовірнісна контекстно-вільна граматика (англ. Probabilistic Context-Free Grammar)

LSTM --- довга короткочасна пам'ятть (англ. Long short-term memory)

LSTM-мережі --- мережі з довгою короткочасною пам'яттю

GAN --- генеративна змагальна мережа (англ. Generative Adversarial Network)

PassGAN --- реалізація GAN мережі для задачі вгадування паролів (англ. Password Generative Adversarial Network)

GNPassGAN --- реалізація GAN мережі для задачі вгадування паролів з використанням градієнтної нормалізації (англ. Gradient Normalization Password Generative Adversarial Network)

Char10 --- паролі довжини до 10 симолів (англ. Characters 10)

Char812 --- паролі довжини від 8 до 12 симолів (англ. Characters 8-12)

FG --- згенерований файл паролів (англ. generated password file)

FT --- тестовний файл паролів (англ. testing password file)